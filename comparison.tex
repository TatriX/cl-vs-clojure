% Created 2016-03-26 Sat 17:48
\documentclass[presentation]{beamer}
\usepackage[utf8]{inputenc}
\usepackage[T1]{fontenc}
\usepackage{fixltx2e}
\usepackage{graphicx}
\usepackage{longtable}
\usepackage{float}
\usepackage{wrapfig}
\usepackage{rotating}
\usepackage[normalem]{ulem}
\usepackage{amsmath}
\usepackage{textcomp}
\usepackage{marvosym}
\usepackage{wasysym}
\usepackage{amssymb}
\usepackage{hyperref}
\tolerance=1000
\usetheme{default}
\author{}
\date{\today}
\title{comparison}
\hypersetup{
  pdfkeywords={},
  pdfsubject={},
  pdfcreator={Emacs 24.5.1 (Org mode 8.2.10)}}
\begin{document}

\maketitle
\begin{frame}{Outline}
\tableofcontents
\end{frame}

\begin{frame}[label=sec-1]{Disclaimer}
First goes vanilla CL then optionally alexandria variant.

Then goes Clojure.
\end{frame}
\begin{frame}[fragile,label=sec-2]{Collect chars from range into vector}
 \begin{verbatim}
(make-array 10 :initial-contents (loop for i from 65 below 75 collect (code-char i)))
(map 'vector #'code-char (iota 10 :start 65))
;; (#\A #\B #\C #\D #\E #\F #\G #\H #\I #\J)
\end{verbatim}

\begin{verbatim}
(vec (map char (range 65 75)))
;; [\A \B \C \D \E \F \G \H \I \J]
\end{verbatim}
\end{frame}
\begin{frame}[fragile,label=sec-3]{Access element of vector}
 \begin{verbatim}
(aref #(1 2 3) 2)
;; 3
\end{verbatim}

\begin{verbatim}
(nth [1 2 3] 2)
(get [1 2 3] 2)
([1 2 3] 2)
;; 3
\end{verbatim}
\end{frame}
\begin{frame}[fragile,label=sec-4]{Vector to list}
 \begin{verbatim}
(coerce #(1 2 3) 'list)
;; (1 2 3)
(reverse (coerce #(1 2 3) 'list))
;; (3 2 1)
\end{verbatim}

\begin{verbatim}
(seq [1 2 3])
;; (1 2 3)

(rseq [1 2 3])
;; (3 2 1)
\end{verbatim}
\end{frame}
\begin{frame}[fragile,label=sec-5]{Skeleton}
\end{frame}
% Emacs 24.5.1 (Org mode 8.2.10)
\end{document}
